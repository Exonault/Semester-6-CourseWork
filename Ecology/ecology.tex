\documentclass[a4paper,fleqn,12pt]{article}
\usepackage[utf8]{inputenc}
\usepackage[bulgarian]{babel}
\usepackage{amsmath}
\usepackage{amssymb}
\usepackage{booktabs}
\usepackage{fancyhdr}
\usepackage{amsthm}
\usepackage{graphicx}

%\pagestyle{fancy}
%\fancyhf{}
%\lhead{\rightmark}
%\rhead{\thepage}
%\cfoot{}
%\renewcommand{\headrulewidth}{0pt}


\begin{document}
\begin{titlepage}
	\setlength{\parindent}{0pt}
	\large
\centering
Технически университет -  София \par
Факултет по приложна математика и информатика \par
\vspace{2cm}

{\huge Курсова работа \par}

\vspace{2cm}

\vspace{1cm}
{\LARGE\scshape  Математическа Екология \par}



\vfill

\begin{minipage}[t]{.5\linewidth}
	Студент: \\
	Кристиян Кръчмаров
\end{minipage}%
\begin{minipage}[t]{.5\linewidth}
	\raggedleft
	Преподавател:\\
	проф. дмн. Людмил Каранджулов
\end{minipage}

\vspace{2cm}
\raggedright

\end{titlepage}
\tableofcontents
\newpage

\section{Задание}
За математическия модел на съжителство на две популации
\begin{gather*}
		\begin{array}{|l@{}} \tag{$\ast$}
		\dot{N_1} = \left(a - bN_1 - \sigma N_2 \right) N_1 \qquad a,b,\sigma > 0\\
		\dot{N_2} = \left(c - \nu N_1 - d N_2 \right) N_2 \qquad c,d,\nu > 0 \\
		\end{array} \\
\text{са въведени следните означения} \\
\Delta = \begin{pmatrix}b & \sigma \\ \nu & d \end{pmatrix} \qquad 
\Delta_1 = \begin{pmatrix}a & \sigma \\ c & d \end{pmatrix} \qquad 
\Delta_2 = \begin{pmatrix}b & a\\ \nu & c \end{pmatrix}
\end{gather*}
Изследвайте вида на особенните точки, 
фазова картина, компютърна реализация, 
съответни чертежи и биологични изводи, ако е изпълнено
	\begin{equation*}
	\Delta > 0 \qquad \Delta_1 > 0 \qquad \Delta_2 > 0
	\end{equation*}

\newpage
\section{Решение}
\subsection{Особенни точки}
Oсобенните точки се получават като решение на системата
\begin{gather*}
		\begin{array}{|l@{}}
		\left(a - bN_1 - \sigma N_2 \right) N_1 = 0\\
		\left(c - \nu N_1 - d N_2 \right) N_2 = 0 \\
		\end{array}
\end{gather*}

\subsubsection{Първи случай}
\begin{gather*}
	\begin{array}{|l@{}}\tag{$I$}
		 N_1 = 0\\
		 N_2 = 0 \\
	\end{array}
\end{gather*}
\subsubsection{Втори случай}
\begin{gather*}
	\begin{array}{|l@{}}\tag{$II$}
		 N_1 = 0\\
		 N_2 \neq 0 \\
	\end{array} \implies c - dN_2 = 0 \implies 
	\begin{array}{|l@{}}
		 N_1 = 0\\
		 N_2 = \frac{c}{d}
	\end{array}
\end{gather*}
\subsubsection{Трети случай}
\begin{gather*}
	\begin{array}{|l@{}}\tag{$III$}
		 N_1 \neq 0\\
		 N_2 = 0 \\
	\end{array} \implies a- bN_1 = 0 \implies 
	\begin{array}{|l@{}}
		 N_1 = \frac{a}{b}\\
		 N_2 = 0
	\end{array} 
\end{gather*}
\subsubsection{Четвърти случай}
\begin{gather*}
	\begin{array}{|l@{}}
		 N_1 \neq 0\\
		 N_2 \neq 0 
	\end{array} \implies 
	\begin{array}{|l@{}}
		a - bN_1 - \sigma N_2 = 0\\
		c - \nu N_1 - d N_2  = 0
	\end{array} \iff 
		\begin{array}{|l@{}}
		 bN_1 + \sigma N_2 = a\\
		\nu N_1 + d N_2  = c
	\end{array}\implies \\
	\begin{array}{|l@{}}\tag{$IV$}
		 N_1 = \frac{\Delta_1}{\Delta} \\
		 N_2 = \frac{\Delta_2}{\Delta}
	\end{array} \qquad \text{(Крамер)} 
\end{gather*}

\subsection{Линеаризация}
Линеаризацията се получава като се замести в ($\ast$)
\begin{gather*}
	\begin{array}{|l@{}}
		 N_1 - \alpha = y_1\\
		 N_2 - \beta = y_2 
	\end{array} \iff
	\begin{array}{|l@{}}
		 N_1 =  y_1 + \alpha\\
		 N_2= y_2 + \beta
	\end{array}
\end{gather*}
където $(\alpha, \beta)$ е особенна точка и се вземе линейната част за всяка една променлива $y_1, y_2$

\begin{gather*}
	W = \begin{pmatrix} a&0\\0&c \end{pmatrix} \\
	W = \begin{pmatrix} a - \frac{\sigma c}{d} & 0\\ - \frac{\nu c}{d}& -c \end{pmatrix} \\
	W = \begin{pmatrix} -a & - \frac{\sigma a}{b}\\0& c - \frac{\nu a}{b} \end{pmatrix} \\
	W = \frac{1}{\Delta} \begin{pmatrix} b\Delta_1 & \sigma\Delta_1 \\ \nu\Delta_2 & d\Delta_2 \end{pmatrix} \text{?!?!}
\end{gather*}
\subsubsection{Първи случай}
\begin{gather*}
	\begin{array}{|l@{}}%\tag{$I$}
		 N_1 - 0 = y_1\\
		 N_2 - 0 = y_2
	\end{array} \iff 
	\begin{array}{|l@{}}%\tag{$I$}
		 N_1 = y_1\\
		 N_2 = y_2
	\end{array} \implies
	\begin{array}{|l@{}}
		\dot{y_1} = \left(a - by_1 - \sigma y_2 \right) y_1\\
		\dot{y_1} = \left(c - \nu y_1 - d y_2 \right) y_2 
	\end{array} \iff \\ \\
	\begin{array}{|l@{}}
		\dot{y_1} = ay_1 - by_1 ^2- \sigma y_1 y_2 \\
		\dot{y_1} = cy_2 - \nu y_1y_2 - d y_2 ^2 
	\end{array} 
\implies  W = \begin{pmatrix} a&0\\0&c \end{pmatrix} \tag{$I$}
\end{gather*}
\subsubsection{Втори случай}
\begin{gather*}
	\begin{array}{|l@{}}%\tag{$I$}
		 N_1 - 0 = y_1\\
		 N_2 - \frac{c}{d}= y_2
	\end{array} \iff 
	\begin{array}{|l@{}}%\tag{$I$}
		 N_1 = y_1\\
		 N_2 = y_2 + \frac{c}{d}
	\end{array} \implies
	\begin{array}{|l@{}}
		\dot{y_1} = \left[a - by_1 - \sigma \left(y_2 + \frac{c}{d} \right) \right] y_1\\
		\dot{y_2} = \left[c - \nu y_1 - d \left(y_2 + \frac{c}{d} \right) \right] \left(y_2 + \frac{c}{d} \right) 
	\end{array} \iff \\ \\
	\begin{array}{|l@{}}
		\dot{y_1} = ay_1 - by_1 ^2- \sigma y_1 y_2 - \frac{\sigma c}{d} y_1\\
		\dot{y_2} = \left[c - \nu y_1 - dy_2 -c \right] \left(y_2 + \frac{c}{d} \right)  
	\end{array} 
\implies  W = \begin{pmatrix} a&0\\0&c \end{pmatrix} \tag{$II$}
\end{gather*}
\subsubsection{Трети случай}
%\begin{gather*}
%	\begin{array}{|l@{}}\tag{$III$}
%		 N_1 \neq 0\\
%		 N_2 = 0 \\
%	\end{array} \implies a- bN_1 = 0 \implies 
%	\begin{array}{|l@{}}
%		 N_1 = \frac{a}{b}\\
%		 N_2 = 0
%	\end{array} 
%\end{gather*}
\subsubsection{Четвърти случай}
%\begin{gather*}
%	\begin{array}{|l@{}}
%		 N_1 \neq 0\\
%		 N_2 \neq 0 
%	\end{array} \implies 
%	\begin{array}{|l@{}}
%		a - bN_1 - \sigma N_2 = 0\\
%		c - \nu N_1 - d N_2  = 0
%	\end{array} \iff 
%		\begin{array}{|l@{}}
%		 bN_1 + \sigma N_2 = a\\
%		\nu N_1 + d N_2  = c
%	\end{array}\implies \\
%	\begin{array}{|l@{}}\tag{$IV$}
%		 N_1 = \frac{\Delta_1}{\Delta} \\
%		 N_2 = \frac{\Delta_2}{\Delta}
%	\end{array} \qquad \text{(Крамер)} 
%\end{gather*}

\subsection{Собствени стойностти}
$$\det(W-\lambda I) = 0$$
%\begin{enumerate}
%    \item[(i)] First item
%    \item[(ii)] Second item
%    \item[(iii)] Third item
%\end{enumerate}











































































\end{document}