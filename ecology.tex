\documentclass[a4paper,fleqn,12pt]{article}
\usepackage[utf8]{inputenc}
\usepackage[bulgarian]{babel}
\usepackage{amsmath}
\usepackage{amssymb}
\usepackage{booktabs}
\usepackage{fancyhdr}
\usepackage{amsthm}
\usepackage{graphicx}

%\pagestyle{fancy}
%\fancyhf{}
%\lhead{\rightmark}
%\rhead{\thepage}
%\cfoot{}
%\renewcommand{\headrulewidth}{0pt}


\begin{document}
\begin{titlepage}
	\setlength{\parindent}{0pt}
	\large
\centering
Технически университет -  София \par
Факултет по приложна математика и информатика \par
\vspace{2cm}

{\huge Курсова работа \par}

\vspace{2cm}

\vspace{1cm}
{\LARGE\scshape  Математическа Екология \par}



\vfill

\begin{minipage}[t]{.5\linewidth}
	Студент: \\
	Кристиян Кръчмаров
\end{minipage}%
\begin{minipage}[t]{.5\linewidth}
	\raggedleft
	Преподавател:\\
	проф. дмн. Людмил Каранджулов
\end{minipage}

\vspace{2cm}
\raggedright

\end{titlepage}
\tableofcontents
\newpage

\section{Задание}
За математическия модел на съжителство на две популации
\begin{gather*}
		\begin{array}{|l@{}} \tag{$\ast$}
		\dot{N_1} = \left(a - bN_1 - \sigma N_2 \right) N_1 \qquad a,b,\sigma > 0\\
		\dot{N_2} = \left(c - \nu N_1 - d N_2 \right) N_2 \qquad c,d,\nu > 0 \\
		\end{array} \\
\text{са въведени следните означения} \\
\Delta = \begin{pmatrix}b & \sigma \\ \nu & d \end{pmatrix} \qquad 
\Delta_1 = \begin{pmatrix}a & \sigma \\ c & d \end{pmatrix} \qquad 
\Delta_2 = \begin{pmatrix}b & a\\ \nu & c \end{pmatrix}
\end{gather*}
Изследвайте вида на особенните точки, 
фазова картина, компютърна реализация, 
съответни чертежи и биологични изводи, ако е изпълнено
	\begin{equation*}
	\Delta > 0 \qquad \Delta_1 > 0 \qquad \Delta_2 > 0
	\end{equation*}
\section{Решение}
\subsection{Особенни точки}
Oсобенните точки се получават като решение на системата
\begin{gather*}
		\begin{array}{|l@{}}
		\left(a - bN_1 - \sigma N_2 \right) N_1 = 0\\
		\left(c - \nu N_1 - d N_2 \right) N_2 = 0 \\
		\end{array}  \\ 
		\text{Решенията са} \\
		\begin{array}{|l@{}}
		 N_1 = 0\\
		 N_2 = 0 \\
		\end{array}
		 \qquad \qquad 
		\begin{array}{|l@{}}
		 N_1 = 0\\
		 N_2 = \frac{c}{d} \\
		\end{array}
		\qquad \qquad 
		\begin{array}{|l@{}}
		 N_1 = \frac{a}{b}\\
		 N_2 = 0 \\
		\end{array}
		\qquad \qquad 
		\begin{array}{|l@{}}
		 N_1 = \frac{\Delta_1}{\Delta} \\
		 N_2 = \frac{\Delta_2}{\Delta} \\
		\end{array}
\end{gather*}











































































\end{document}